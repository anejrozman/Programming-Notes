\documentclass[12pt, a4paper]{article}
\usepackage[slovene]{babel}
\usepackage[T1]{fontenc}
\usepackage[utf8]{inputenc}
\usepackage{amsmath,amssymb,amsfonts,amsthm}
\usepackage{url}
\usepackage[dvipsnames,usenames]{color}
\usepackage{graphicx}
\usepackage{tikz}
\usepackage{dsfont}
\usepackage{caption}
\usepackage{subcaption}
\usepackage{bm}
\usepackage{float}
\usepackage{xcolor}
\usepackage{bbm}
\usepackage{makeidx}
\usepackage[
    colorlinks=true,
    linkcolor=violet,      % Color of internal links (sections, pages, etc.)
    citecolor=teal,    % Color of citation links
    filecolor=magenta,  % Color of file links
    urlcolor=purple       % Color of external links (URLs)
]{hyperref}
%\usepackage{titlesec}
\allowdisplaybreaks
\makeindex

%--------------------------------------OBLIKA DOKUMENTA--------------------------------------------%

% Oblika strani
\textwidth 15cm
\textheight 24cm
\oddsidemargin.5cm
\evensidemargin.5cm
\topmargin-5mm
\addtolength{\footskip}{10pt}
\pagestyle{plain}
\overfullrule=15pt % oznaci predlogo vrstico

% Naslovi
%\titleformat{\section}
%{\normalfont\Large\bfseries\centering\MakeUppercase}{\thesection}{1em}{}
%
%\titleformat{\subsection}
%{\normalfont\large\bfseries}{\thesubsection}{1em}{}
%
%\titleformat{\subsubsection}
%{\normalfont\normalsize\bfseries}{\thesubsubsection}{1em}{}


% Matematicna okolja (tekst napisan pokoncno)
\theoremstyle{definition}
\newtheorem{definicija}{Definicija}[section]
\newtheorem{zgled}[definicija]{Zgled}
\newtheorem{opomba}[definicija]{Opomba}

\renewcommand\endzgled{\hfill$\diamondsuit$}

% Matematicna okolja (tekst napisan posevno)
\theoremstyle{plain} 
\newtheorem{lema}[definicija]{Lema}
\newtheorem{izrek}[definicija]{Izrek}
\newtheorem{trditev}[definicija]{Trditev}
\newtheorem{posledica}[definicija]{Posledica}



% Ukaz za slovarsko geslo
\newlength{\odstavek}
\setlength{\odstavek}{\parindent}
\newcommand{\geslo}[2]{\noindent\textbf{#1}\hspace*{3mm}\hangindent=\parindent\hangafter=1 #2}

% Podatki o delu
\newcommand{\imeavtorja}{Anej Rozman} 
\newcommand{\naslovdela}{KAJ JE PROGRAM?} 
\newcommand{\letnica}{2941} 

%-----------------------------------------DODATNI UKAZI--------------------------------------------%
\newcommand{\R}{\mathbb{R}}
\newcommand{\N}{\mathbb{N}}
\newcommand{\E}{\mathbb{E}}
\newcommand{\F}{\mathcal{F}}
\newcommand{\B}{\mathcal{B}}
\newcommand{\1}{\mathds{1}}



%------------------------------------------NASLOVNE STRANI-----------------------------------------%

\begin{document}

\thispagestyle{empty}
\vfill

\begin{center}{\large}
\imeavtorja\\[2mm]
{\bf \naslovdela}\\[10mm]
Osebni zapiski sintakse jezikov \textit{Python, C++} in \textit{OCaml} ter nekaterih 
(mogo"ce uporabnih) algoritmov\\[1cm]
\end{center}
\vfill

\noindent{\large
London, \letnica}
\pagebreak

\thispagestyle{empty}
\tableofcontents
\pagebreak


%---------------------------------------------PYTHON-----------------------------------------------%
\section{Python}
Samo test "ce stvarno \index{Python} deluje.



















%---------------------------------------------C++--------------------------------------------------%
\pagebreak
\section{C++}






















%---------------------------------------------OCAML------------------------------------------------%
\pagebreak
\section{OCaml}




























%---------------------------------------------PSA--------------------------------------------------%
\pagebreak
\section{Podatkovne strukture in algoritmi}







%------------------------------------------STVARNO KAZALO------------------------------------------%
\pagebreak
\printindex
\end{document}